\documentclass[a4paper,12pt]{article}
\usepackage[brazil]{babel}
\usepackage[utf8]{inputenc}
\newcommand{\TOP}{\top}
\newcommand{\BOT}{\bot}
 \newcommand{\NOT}{\neg}
\newcommand{\CONS}{\circ}
\newcommand{\INCONS}{\bullet}
\newcommand{\IMP}{\to}
 \newcommand{\AND}{\wedge}
\newcommand{\OR}{\vee}
\newcommand{\BIMPLI}{\leftrightarrow}
\newcommand{\XOR}{\otimes}
 \newcommand{\T}{\begin{tt}T\end{tt}\, }
\newcommand{\F}{\begin{tt}F\end{tt}\, }
 \newcommand{\x}{\ensuremath{\times}}
\newcommand{\BIGOR}{\bigvee}		
\newcommand{\BIGAND}{\bigwedge}	
 \newcommand{\CR}{\vdash}
\newcommand{\SR}{\models}
\newcommand{\EQUALDEF}{\stackrel{\mbox{\tiny def}}{=}}

\begin{document}

\begin{center}	
	{\LARGE Apostila de Exercícios:}
\end{center}

\begin{center}
\section{Exercícios de Conjuntos}
\end{end}

\begin{enumerate}
\item Sendo o conjunto A = \{4,  6,  9, 2, 11\} e B = \{2, 5, 2, 9\}, resolva as seguintes operações sobre conjuntos:

\begin{enumerate} \item $A \cup B$ \\ \item $B \cap A$ \\ \item $B - A$ \\ \item $A \x B$ \\

\end{enumerate}


\item Demonstre o Diagrama de Venn dos seguintes casos:

\begin{enumerate} \item E = \{6, 8, 4, 2\},  F = \{b, d, h, f\} \item A = \{gato, pato, urso\}, B = \{carne, ração, milho\}

\end{enumerate}

\item De acordo com os conjuntos A e B do exercício 1, responda se as seguintes afirmações são verdadeiras ou falsas:

\begin{enumerate} \item A ⊂ B \item B C A \item A ⊆ B

\end{enumerate}

\item Demostre o conjunto das partes de cada um dos seguintes conjuntos:

 \begin{enumerate} \item A = \{5, 7, 2\},     P(A) = ? \item B = \{6, 1, 5\},    P(B) = ? \item C = \{23, 54, 67, 123\}, P(C) = ? \item D = \{6, 5, 4, 3, 2\},      P(D) = ?
\end{enumerate}

\item (OSEC) Numa escola de 360 alunos, onde as únicas matérias dadas são matemática e português, 240 alunos estudam matemática e 180 alunos estudam português. O número de alunos que estudam matemática e português é: 

\begin{enumerate} \item 120 \item 60 \item 90 \item 180 \item N.d.a. 

\end{enumerate}

\item (PUC-CAMPINAS) Numa indústria, 120 operários trabalham de manhã, 130 trabalham à tarde, 80 trabalham à noite; 60 trabalham de manhã e à tarde, 50 trabalham de manhã e a noite, 40 trabalham à tarde e à noite e 20 trabalham nos três períodos. Assim: 

\begin{enumerate}
\item 150 operários trabalham em 2 períodos; 
\item Há 500 operários na indústria; 
\item 300 operários não trabalham à tarde; 
\item Há 30 operários que trabalham só de manhã; 
\item N.d.a;

\end{enumerate}

\item (NUNO LISBOA) Um subconjunto X de números naturais contém 12 múltiplos de 4, 7 múltiplos de 6, 5 múltiplos de 12 e 8 números ímpares. O número de elementos de X é: 

\begin{enumerate}

\item 22 
\item 27 
\item 24 
\item 32 
\item 20 

\end{enumerate}

\item (CESGRANRIO) Em uma universidade são lidos dois jornais A e B; exatamente 80\% dos alunos lêem o jornal A e 60\% o jornal B. Sabendo-se que todo aluno é leitor de pelo menos um dos jornais, o percentual de alunos que lêem ambos é: 

\begin{enumerate}

\item 48\% 
\item 60\% 
\item 40\% 
\item 140\% 
\item 80\%

\end{enumerate}

\item Em uma pesquisa de mercado foram entrevistadas várias pessoas acerca de suas preferências em relação a três produtos A, B e C. Os resultados da pesquisa indicaram que: 
  
\begin{enumerate}

210 compram o produto A. \\
210 compram o produto B. \\
250 compram o produto C. \\
20 compram os três produtos. \\ 
100 não compram nenhum dos três produtos. \\ 
60 compram os produtos A e B. \\
70 compram os produtos A e C. \\
50 compram os produtos B e C. \\

Quantas pessoas foram entrevistadas? 
\end{enumerate}

\item (UFBH) Um colégio ofereceu cursos de inglês e francês, devendo os alunos se matricularem em pelo menos um deles. Dos 45 alunos de uma classe, 13 resolveram estudar tanto inglês quanto francês; em francês, matricularam-se 22 alunos. Quantos alunos se matricularam em inglês? 


\item (FAAP) Os sócios dos clubes A e B formam um total de 2200 pessoas. Qual é o número de sócios do clube B se A tem 1600 e existem 600 que pertencem aos dois clubes? 


\item (MED. RIO PRETO) Num almoço, foram servidos, entre outros pratos, frangos e leitões. Sabendo-se que, das 94 pessoas presente, 56 comeram frango, 41 comeram leitão e 21 comeram dos dois, o número de pessoas que não comeram nem frango nem leitão é: 

\begin{enumerate}
\item 10 
\item 12 
\item 15 
\item 17 
\item 18 
\end{enumerate}

\item(UFPA) Uma escola tem 20 professores, dos quais 10 ensinam Matemática, 9 ensinam Física, 7 Química e 4 ensinam Matemática e Física. Nenhum deles ensina Matemática e Química. Quantos professores ensinam Química e Física e quantos ensinam somente Física?

\end{enumerate}
\newpage

\begin{center}
\section{Exercícios de Lógica}
\end{center}

\begin{enumerate}\item Sejam as proposições: 
p : está frio 
q : está chovendo
Traduzir para a linguagem natural as seguintes proposições:

\begin{enumerate}

\item $\NOT p$ 
\item $p \AND q$ 
\item $p \OR q$
\item $q \BIMPLI p$ 
\item $p \IMP\NOT q$ 
\item $p \OR\NOT q$
\item $\NOT p \AND\NOT q$
\item $p \BIMPLI \NOT q$
\item $p \AND\NOT q \IMP p$

\end{enumerate}

\item Construir a tabela-verdade para a proposição: $p \OR\NOT q$


\item Sabendo-se que V(p) = V(q) = T (verdadeiro) e V(r) = V(s) = F (falso), determine os valores lógicos das seguintes proposições:

\begin{enumerate}

\item  $(p \AND (q \OR r)) \IMP (p \IMP (r \OR q))$
\item  $(q \IMP r) \BIMPLI (\NOT q \OR r)$
\item  $(\NOT p  \OR\NOT(r \AND s))$
\item  $\NOT(q \BIMPLI ( \NOT p \AND s))$
\item  $(p \BIMPLI q) \OR (q \IMP\NOT p)$
\item  $\NOT(\NOT q \AND (p \AND\NOT s))$
\item  $\NOT q \AND ((\NOT r \OR s) \BIMPLI (p \IMP \NOT q))$
\item  $\NOT(\NOT p \OR (q \AND s)) \IMP (r \IMP \NOT s)$
\item  $\NOT (p \IMP (q \IMP r)) \IMP s$

\end{enumerate}

 \item Construir as tabelas verdade para as seguintes proposições:

\begin{enumerate}

\item  $ p \OR \NOT r \IMP q \OR \NOT r $
\item  $ \NOT(p \AND q) \OR \NOT(q \BIMPLI p)$
\item  $(p \AND q \IMP r) \OR (\NOT p \BIMPLI q \OR \NOT r)$

\end{enumerate}


\item Aplicando as Leis de Morgan, dar a negação de cada uma das seguintes proposições:

\begin{enumerate}

\item $p \AND \NOT q$
\item $\NOT p \AND\NOT q$
\item $\NOT p \OR q$
\item $\NOT p  \OR\NOT q$

\end{enumerate}

\item Simplifique as proposições abaixo utilizando as leis de equivalência. (indique qual lei você está usando durante a simplificação).

\begin{enumerate}

\item $p \AND (p \IMP q) \AND (p \IMP\NOT q)$
\item $(p \IMP q) \AND (\NOT p \IMP q)$
\item $(p \IMP q) \IMP r$
\item $(p \AND q) \IMP (\NOT r \IMP \NOT q)$
\item $p \IMP (p \OR q)$
\item $p \BIMPLI q$

\end{enumerate} \end{enumerate}



\newpage
\begin{center}
\section{Exercícios de Pré Cálculo}
\end{center}
\begin{enumerate} \item Fatore os termos, colocando-os em evidência:

\begin{enumerate} 

\item $ x^2 - xy $
\item $ a^3 - a^2b $
\item $ 6x^3 - 12x^2 + 36 $
\item $ 12x^3y^4 - 18x^2y^3 + 6x^4y $
\item $ 32m^7 p^1^0 + 95m^5 p^6 - 128m^4p^8 $

\end{enumerate}

\item Fatore as expressões algébricas:

\begin{enumerate}

\item $ 16x^4 - 1 $
\item $ x^3 - 6x^2 + 12x - 8 $
\item $ a^1^2 - a”^2^0 $
\item $ x^2 - (a+1)^2 $
\item $ m^8 - y^8 $

\end{enumerate}

\item Simplifique ao máximo as seguintes expressões:

\begin{enumerate}
\item $\frac{(x + xy)}{(x + xz)}$
\item $\frac{(x^2 - 9)}{(x - 3)}$
\item $\frac{(x^2 - 5x + xy + 5y)}{(7x + 7y)}$
\item $\frac{(4x^2 + 4xy + y^2)}{(2x + y)}$
\item $\frac{[a^2 + ab + (b + a)(b - a)]}{(3a + 3b)}$
\item $\frac{[(x + y)(x + y) - y^2]}{(x + 2y)}$
\end{enumerate}

\item Resolva as seguintes inequações de $1º$ grau:

\begin{enumerate}
\item $ 2x + 5 < -3x + 40 $
\item $ 6(x - 5) - 2(4x + 2) > 100  $
\item $ 7x - 9 < 2x + 16$
\item $ -(8 - 4x - 7) \leq 2x + 7 $
\end{enumerate}

\item Demonstre o conjunto solução das seguintes inequeações do $2º$ grau:

\begin{enumerate}
\item $x^2 - 5x + 6 > 0$
\item $x^2 + x - 12 \leq 0$
\item $-x^2 + 6x - 8 > 0$
\end{enumerate}


\end{document}
